%! TeX program = lualatex
\documentclass[11pt]{book}
\usepackage{../MathLetter-revised}
\usepackage[italicdiff]{physics}

\setcounter{day}{1}
\setcounter{page}{1}
\fontsettingtrue
\mergedcountertrue

\title{처음 접하는 이상한 적분의 계산법}
\author{KAIST 물리학과 21학번 이종민}
\date{January 2, 2023}

\begin{document}
\maketitle[Lecture Note]
\section*{서론}
\begin{MLPar}
고등학교에 들어온 후로 많은 적분 계산을 해봤을 것입니다.
적분도 모르는 채로 입학해서 물리학 I 시간에 Gauss 법칙으로 이를 처음 접한 학생들도 분명히 있을 거예요.
하지만 통상적으로 고등학교까지 배우는 수학, 즉 중등 교육과정 내에서는 아주 간단한 형태의 적분을 만나게 됩니다.
2학년을 갓 마친 친구들은 Calculus 수업 시간에 극좌표계나 다변수 적분 등도 계산해봤을 것이고요.

그런데 대학에 입학하면, 물리학 및 공학 분야에서는 더 다양하고 많은 수학을 요구하고, 배우게 됩니다.
새내기 때 미적분학을 듣고, 다음 해에 수리물리학이나 공업수학과 같은 이름을 가진 과목에서 한 해 동안 수학을 더 배우는 것이지요.
물론 다른 주제들도 많이 다루지만, 이번 강의에서는 언급한 과목들에서 처음 배우는 적분을 다뤄보고자 합니다.
구체적으로, 복소함수를 이용하는 적분과, 라이프니츠 규칙을 이용하는 적분을 각각 다룰 것입니다.

복소함수를 이용한 적분은 \newterm{유수정리}[residue theorem]를 이용합니다.
\begin{equation*}
\int_{-\infty}^{\infty}\dd{x} \frac{e^{ax}}{e^x + 1} \quad(0 < a < 1)
\end{equation*}
과 같은 상상이 안 되는 적분을 셈할 수 있게 됩니다.

라이프니츠 규칙을 이용하는 적분은 다른 이름으로 파인만의 트릭이라고도 부릅니다.
이를 이용하면 양자역학 등에서 자주 사용하는
\begin{equation*}
    \int_{-\infty}^{\infty}\dd{x} x^{2n} e^{-x^2} \quad\text{for}\quad n=0,\ 1,\ 2, \ldots
\end{equation*}
와 같은 적분을 계산할 수 있게 됩니다!
위 적분 문제는 직접 풀어볼 것이니, 조금만 기다려봅시다.
물론 수리과학을 전공하고 싶어하는 학생의 경우, 증명 과정이 어설퍼보일 수 있지만, 양해 부탁드립니다.
\end{MLPar}

\section{복소함수}
\begin{MLPar}
내용
\end{MLPar}
% Branch cut (가지 절단)

\section{라이프니츠 규칙}
\begin{MLPar}
내용
\end{MLPar}

\end{document}